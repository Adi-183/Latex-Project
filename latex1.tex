\documentclass{article}
\usepackage[utf8]{inputenc}
\usepackage{amsmath}
\newcommand\tab[1][1cm]{\hspace*{#1}}
%this is a comment

\title{Hello World!}

\author{Aditya Patel}
\date{November 16, 2021}
\begin{document}
\pagenumbering{gobble}

\maketitle

\section{Getting Started}
\textbf{Hello World!} Today I am learning \LaTeX . \LaTeX is a great program for writing math. I can write line math such as $ a^{2}+b^{2}=c^{2} $. I can also give equations their own space: 
\begin{equation}
   \gamma^{2} + \theta^{2}=\omega^{2}    
\end{equation}
"Maxwell" equations are named for James Clark Maxwell and are as follows:\\
\begin{equation}
   \tab \:\: \: \Vec{ \nabla}.\Vec{E}  =\frac{\rho}{\epsilon_0} \tab \tab \tab    \text{Gauss's Law} \tab \tab \tab \:\:\:
\end{equation}
\begin{equation}
    \Vec{ \nabla}.\Vec{B}  =0 \tab \tab \tab \: \: \text{Gauss's Law for Magnetism}
\end{equation}
\begin{equation}
    \Vec{ \nabla}\times\Vec{E}  =-\frac{\partial \Vec{B}}{\partial t} \tab \tab \: \: \:  \text{Faraday's Law of Induction} 
\end{equation}
\begin{equation}
    \tab  \quad \Vec{ \nabla}\times\Vec{B}=\mu_0 \left( \epsilon_0\frac{\partial \Vec{E}}{\partial t} + \Vec{J} \right)  \quad  \: \text{Ampere's Circuital Law} \tab  \quad \:
\end{equation}
Equations 2,3,4 and 5 are some of the most important in Physics.
\section{What about Matrix Equations?}
\[
\begin{pmatrix}
a_{11}&a_{12}&\cdots&a_{1n}\\
a_{21}&a_{22}&\cdots&a_{2n}\\
\vdots&\vdots&\ddots&\vdots \\
a_{n1}&a_{n2}&\cdots&a_{nn}\\
\end{pmatrix}
\begin{bmatrix}
v_1\\
v_2\\
\vdots\\
v_n\\
\end{bmatrix}
=
\begin{matrix}
w_1\\
w_2\\
\vdots\\
w_n\\
\end{matrix}
\]
\end{document}
